\documentclass[a4paper]{article}
\usepackage[spanish,es-tabla]{babel}	% trabajar en español
\spanishsignitems	
%\usepackage{simplemargins}

%\usepackage[square]{natbib}
\usepackage{amsmath}
\usepackage{amsfonts}
\usepackage{amssymb}
\usepackage{graphicx}
\usepackage{blindtext}
\usepackage{hyperref}

\begin{document}
\pagenumbering{gobble}

\Large
 \begin{center}
Reporte de Seminario 1\\
Computación Cuántica  

\hspace{10pt}

% Author names and affiliations
\large
Lic. Julio A. Medina$^1$ \\

\hspace{10pt}
\small  
$^1$ Universidad de San Carlos, Escuela de Ciencias Físicas y Matemáticas\\
Maestría en Física\\
\href{mailto:julioantonio.medina@gmail.com}{julioantonio.medina@gmail.com}\\

\end{center}

\hspace{10pt}

\normalsize
Durante el curso de Seminario 1 se han investigado los fundamentos teóricos de la computación cuántica. Se han repasado temas de mecánica estadística, como el concepto de ensambles, un tema recurrente en computación cuántica. La computación cuántica nace del deseo de Richcard P. Feynman de querer analizar y simular sistemas cuánticos por medio de sistemas informáticos, esto llevo rápidamente a la realización que el problema crecía exponencialmente con respecto al número de partículas a simular, esto implicaba que el costo computacional crecía de la misma manera, esto imposibilitó de gran manera simular dichos sistemas cuánticos. Feynman\cite{Feynman} publicó en 1981 un artículo seminal en el que plantó la semilla para crear sistemas de computación cuántica para simular mas efectivamente sistemas de naturaleza cuántica. \\

Se han revisado conceptos fundamentales y básicos de la computación cuántica como:
\begin{itemize}
\item El concepto del bit cuántico, \textbf{\textit{qubit}}.
\item Entrelazamiento cuántico.
\item Superposición de estados cuánticos.
\item Producto tensorial.
\item La esfera de Bloch.
\item Compuertas lógicas cuánticas.
\item Operador de Hadamard.
\item Teletransportación cuántica.
\item Transformada cuántica de Fourier.
\item Circuitos cuánticos.
\item Algoritmo de Shor.

\end{itemize} 
Ademas de la investigación teórica se han aprendido conceptos prácticos a través de la API para computación cuántica creada por IBM, Qiskit \cite{Qiskit}. Dicha API se usa como una biblioteca de python mediante la cual se pueden accesar a computadoras cuánticas reales en las instalaciones de IBM para poder probar y desarrollar circuitos cuánticos básicos. 

%En este trabajo de graduación consta de dos partes, en la primera parte se ha realizado una investigación sobre el enfoque de la mecánica estadística en la redes neuronales. En la segunda parte
\begin{thebibliography}{99}
%% La bibliografía se ordena en orden alfabético respecto al apellido del 
%% autor o autor principal
%% cada entrada tiene su formatado dependiendo si es libro, artículo,
%% tesis, contenido en la web, etc


\bibitem{Nielsen} Michael A.Nielsen, Isaac L. Chuang \textit{Quantum Computation adn Quantum Information}. Cambridge University Press 2010. 10th. Anniversary Edition.

\bibitem{Feynman} Richard P. Feynman. \textit{Simulating Physics with Computers.} \url{https://doi.org/10.1007/BF02650179}.

\bibitem{Qiskit} \textit{Qiskit Textbook}. \url{https://qiskit.org/textbook-beta}

\bibitem{Mermin} N. David Mermin \textit{Quantum Computer Science: An Introduction}. Cambridge University Press, 2007.

\bibitem{Sakurai} J.J. Sakurai \textit{Modern Quantum Mechanics}. The Benjamin/Cummings Publishing Company, 1985.

\bibitem{Dotsenko} Viktor Dotsenko. \textit{An Introduction to the Theory of Spin Glasses and Neural Networks}. World Scientific 1994.

\bibitem{Bahri} Yasaman Bahri, Jonathan Kadmon, Jeffrey Pennington, Sam S. Schoenholz, Jascha Sohl-Dickstein, Surya Ganguli. \textit{Statistical Mechanics of Deep Learning}. \url{https://www.annualreviews.org/doi/pdf/10.1146/annurev-conmatphys-031119-050745}

\bibitem{Hopfield} J.J. Hopfield. \textit{Neural Networks and physical systems with emergent collective computational abilities}. \url{https://doi.org/10.1073/pnas.79.8.2554}


\bibitem{McCulloch} Warren S. McChulloch, Walter H. Pitts. \textit{A LOGICAL CALCULUS OF THE IDEAS IMMANENT IN NERVOUS ACTIVITY}. \url{http://www.cse.chalmers.se/~coquand/AUTOMATA/mcp.pdf}



\end{thebibliography}
\end{document}

